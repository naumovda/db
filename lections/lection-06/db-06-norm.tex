\documentclass{beamer}
\mode<presentation>
\usetheme{CambridgeUS}
\usepackage[russian]{babel}
\usepackage[utf8]{inputenc}
\usepackage[T2A]{fontenc}
\usepackage{sansmathaccent}

\usepackage{verbatim}
\usepackage{alltt}

\pdfmapfile{+sansmathaccent.map}
\title[СУБД]{Нормализация отношений}
\author{Наумов Д.А., доц. каф. КТ}
\date[14.03.2020] {Базы данных и базы знаний, 2020}

\begin{document}

%ТИТУЛЬНЫЙ СЛАЙД
\begin{frame}
  \titlepage
\end{frame}
  
%СОДЕРЖАНИЕ ЛЕКЦИИ
\begin{frame}
  \frametitle{Содержание лекции}
  \tableofcontents  
\end{frame}
  
%РАЗДЕЛ 1
\section{Типы приложений: транзакционная и аналитическая обработка}
\begin{frame}
OLTP (On-Line Transaction Processing) — интерактивная транзакционная обработка.
\begin{itemize}
\item запросы пользователей выполняются одновременно - 
обработка идёт в режиме реального или приближенного к реальному времени;
\item запросы представляют собой интенсивный поток коротких операций по вставке, изменению и удалению небольшого числа записей в БД;
\item большая часть запросов известна на этапе проектирования;
\item время выполнения сложных аналитических запросов не является критическим для системы;
\end{itemize}
Примеры OLTP-приложений:
\begin{itemize}
\item системы складского учета;
\item системы заказа билетов;
\item банковские системы;
\end{itemize}
Данные OLTP-приложений \textbf{сильно нормализованы}.
\end{frame}

\begin{frame}
OLAP (On-Line Analytical Processing) - интерактивная аналитическая обработка. 
\begin{itemize}
\item Данные находятся \textbf{в режиме чтения}, за исключением моментов их обновления.
\item Выборки представляют собой \textbf{одиночные тяжёлые запросы}: поиски и расчёты по множеству \textbf{произвольных} критериев.
\item \textbf{Время отклика системы не регламентировано}.
\item Размеры базы данных на порядок и больше транзакционных.
\end{itemize}
\begin{block}{Пример типовой архитектуры OLAP}
\begin{center}
\includegraphics[scale=0.5]{images/olap.png}
\end{center}
\end{block}
\end{frame}

\section{Нормализация}
\begin{frame}
\begin{block}{Нормализация}
метод создания набора отношений с заданными свойствами на основе требований к данным, установленными в конкретной предметной области.
\end{block}
Цели нормализации:
\begin{itemize}
\item устранение избыточности при хранении данных, приводящей к увеличению размера БД.
\item исключение необходимости модификации данных в связных таблицах для минимизации времени и операций, проводящихся в одной транзакции.
\end{itemize}
Процесс нормализации:
\begin{enumerate}
\item исходная точка - представление предметной области в виде одного (или нескольких) отношений
\item на каждом шаге создается набор схем отношений, обладающих лучшими свойствами;
\item критерий <<лучше-хуже>> записит от целей проектирования.
\end{enumerate}
\end{frame}

\begin{frame}{Критерии оценки качества логичекой модели}
\begin{block}{Критерии оценки качества логичекой модели}
\begin{enumerate}
\item Адекватность базы данных предметной области.
\item Скорость выполнения операций обновления данных (вставка, обновление, удаление кортежей).
\item Скорость выполнения операций выборки данных.
\item Легкость разработки и сопровождения базы данных.
\item Отсутствие неоправданной избыточности данных.
\end{enumerate}
\end{block}
Критерии OLAP и OLTP:
\begin{itemize}
\item OLAP: на первый план выходит время отклика системы, данные могут быть избыточны.
\item OLTP: на первый план выходит обработка транзакций в режиме реального времени.
\end{itemize}
\end{frame}

\begin{frame}
\begin{block}{Избыточность данных}
одни и те же факты можно многократно получить из разных объектов базы данных
\end{block}
\begin{itemize}
\item польза: возможностm ускорения выполнения запросов;
\item избыточность должна быть контролируемой: необходима программная реализация проверок того, что избыточные и базовые данные адекватно согласованы между собой;
\end{itemize}
Расписание:
\begin{center}
\includegraphics[scale=0.45]{images/ex-rasp-01.png}
\end{center}
Должности:
\begin{center}
\includegraphics[scale=0.45]{images/ex-rasp-02.png}
\end{center}
\end{frame}

\begin{frame}
\begin{block}{Основные свойства нормальных форм}
\begin{itemize}
\item каждая следующая нормальная форма в некотором смысле улучшает свойства предыдущей;
\item при переходе к следующей нормальной форме свойства предыдущих нормальных форм сохраняются;
\end{itemize}
\end{block}
\begin{center}
\includegraphics[scale=0.75]{images/forms.png}
\end{center}
\end{frame} 

\begin{frame}
\begin{block}{Схемы БД называются эквивалентными}
если содержание исходной БД может быть получено путем эквивалентного соединения отношений, входящих в результирующую схему, и при этом не появляется новых кортежей.
\end{block}
Исходная схема отношения:
\begin{center}
\includegraphics[scale=1.0]{images/ex-rasp-03.png}
\end{center}
Эквивалентная схема, полученная путем декомпозиции на два отношения:
\begin{center}
\includegraphics[scale=1.2]{images/ex-rasp-04.png}
\end{center}
\begin{center}
\includegraphics[scale=1.2]{images/ex-rasp-05.png}
\end{center}
\end{frame} 

\begin{frame}
\begin{block}{1НФ - первая нормальная форма}
выполняется, если все значения атрибутов (колонок таблицы) атомарны, то есть неделимы.
\end{block}
\begin{itemize}
\item cобственные типы данных СУБД считаются атомарными, исключение
могут составлять массивы, в том числе символьные (текстовые) и байтовые.
\item атомарность может быть относительна выбранного взгляда со стороны предметной области и контекста. 
\end{itemize}
Примеры:
\begin{itemize}
\item телефонный номер (в базе данных маркетинга, у телефонных операторов);
\item колонки для хранения комментариев;
\item целая и дробная части действительного числа, дата-время;
\item фамилия, имя, отчество в одной колонке.
\end{itemize}
\begin{center}
\includegraphics[scale=1.5]{images/ex-rasp-06.png}
\end{center}
\end{frame} 

\begin{frame}
\begin{block}{Функциональная зависимость}
Пусть R является переменной отношения, а X и Y — произвольными подмножествами множества атрибутов переменной отношения R. 
\[X\rightarrow Y\]
Y функционально зависимо от X тогда и только тогда, когда для любого допустимого значения переменной отношения R, если два кортежа переменной отношения R совпадают по значению X, они также совпадают и по значению Y.
\end{block}
\begin{itemize}
\item Подмножество X - \textbf{детерминант}, а Y - \textbf{зависимая часть}.
\item Функциональная зависимость \textbf{тривиальна} тогда и только тогда, когда её правая (зависимая) часть является подмножеством её левой части (детерминанта).
\item Функциональная зависимость называется \textbf{неприводимой слева}, если ни один атрибут не может быть опущен из её детерминанта без нарушения зависимости. 
\end{itemize}
\end{frame}

\begin{frame}
\begin{block}{Переменная отношения находится в 2НФ}
тогда и только тогда, когда она находится в первой нормальной форме и каждый неключевой атрибут неприводимо зависит от (каждого) её потенциального ключа
\end{block}
\begin{itemize}
\item \textbf{Неприводимость}: в составе потенциального ключа отсутствует меньшее подмножество атрибутов, от которого можно также вывести данную функциональную зависимость.
\item Если потенциальный ключ является составным, то в отношении не должно быть неключевых атрибутов, зависящих от части составного потенциального ключа. 
\end{itemize}
\begin{center}
\includegraphics[scale=1.4]{images/ex-rasp-07.png}
\end{center}
\end{frame}

\begin{frame}
\textbf{Ставки}(\underline{Должность, Год}, Ставка, Чтение лекций)
\begin{center}
\includegraphics[scale=1.25]{images/ex-rasp-07.png}
\end{center}
Ключи и функциональные зависимости:
\begin{itemize}
\item ключ: \underline{Должность, Год}
\item функциональная зависимость: Должность, Год $\rightarrow$ Ставка
\item функциональная зависимость: Должность $\rightarrow$ Чтение лекций
\end{itemize}
Нарушение второй нормальной формы: атрибут функционально зависит от части первичного ключа.

Декомпозиция:
\begin{itemize}
\item отношение \textbf{Ставки}(\underline{Должность, Год}, Ставка)
\item отношение \textbf{Должности}(\underline{Должность}, Чтение лекций)
\end{itemize}
\end{frame}

\begin{frame}
\begin{block}{Переменная отношения находится в 3НФ}
тогда и только тогда, когда для каждой из её функциональных зависимостей X $\rightarrow$ A выполняется хотя бы одно из следующих условий: 
\begin{itemize}
\item Х содержит А (то есть X $\rightarrow$ A - тривиальная функциональная зависимость);
\item Х - суперключ (не существует двух кортежей, в которых значения атрибутов X совпадают);
\item А - ключевой атрибут (А входит в состав потенциального ключа).
\end{itemize}
\end{block}
Пример: продажа каждой товарной позиции имеет своим основанием документ (заказ, счёт и т.д.), а её стоимость характеризуется ценой, количеством и валютой.

Транзитивные зависимости:
\begin{itemize}
\item Идентификатор продажи $\rightarrow$ Номер документа
\item Идентификатор продажи $\rightarrow$ Код валюты
\item Номер документа $\rightarrow$ Код валюты
\end{itemize}
\end{frame}

\begin{frame}
Пример: \textbf{Студент}(\underline{№ зачетки}, ФИО, Группа, Факультет, Специальность, Выпускающая кафедра)
\begin{block}{Функциональные зависимости}
\begin{itemize}
\item № зачетки $\rightarrow$ ФИО, Группа, Факультет, Специальность, Выпускающая кафедра
\item Группа $\rightarrow$ Факультет, Специальность, Выпускающая кафедра
\item Выпускающая кафедра $\rightarrow$ Факультет
\end{itemize}
\end{block}
\begin{block}{Итоговые отношения}  
\begin{itemize}
\item \textbf{Студенты} (\underline{№ зачетки}, ФИО, Группа)
\item \textbf{Группы} (\underline{Группа}, Специальность, Выпускающая кафедра)
\item \textbf{Выпускающие кафедры} (\underline{Выпускающая кафедра}, Факультет)
\end{itemize}
\end{block}
\end{frame}

\begin{frame}
\begin{block}{Переменная отношения находится в НФБК}
тогда и только тогда, когда каждая её нетривиальная и неприводимая слева функциональная зависимость имеет в качестве своего детерминанта некоторый потенциальный ключ
\begin{itemize}
\item НФБК - нормальная форма Бойса-Кодда;
\item менее строгое определение: детерминанты всех её функциональных зависимостей являются потенциальными ключами.
\end{itemize}
\end{block}
Ситуация, когда отношение будет находиться в 3НФ, но не в НФБК:
\begin{itemize}
\item отношение имеет два (или более) потенциальных ключа, которые являются составными
\item между отдельными атрибутами таких ключей существует функциональная зависимость
\end{itemize}
Поскольку описанная зависимость не является транзитивной, то такая ситуация под определение 3НФ не подпадает. 
\end{frame}

\begin{frame}
Пример: Планирование занятий
\begin{center}
\includegraphics[scale=0.5]{images/ex-rasp-28.png}
\end{center}
Свойства отношения:
\begin{itemize}
\item все атрибуты входят в какой-то из потенциальных ключей, неключевых атрибутов в отношении нет (2НФ).
\item нет транзитивных зависимостей (3НФ).
\item функциональная зависимость Планирование занятий $\rightarrow$ Аудитория, в которой левая часть (детерминант) не является потенциальным ключом отношения, то есть отношение не находится в нормальной форме Бойса — Кодда.
\end{itemize}
\end{frame}

\begin{frame}
Исходное отношение:
\begin{center}
\includegraphics[scale=0.5]{images/ex-rasp-28.png}
\end{center}
Нормализованные отношения:
\begin{center}
\includegraphics[scale=0.5]{images/ex-rasp-29.png}
\includegraphics[scale=0.5]{images/ex-rasp-30.png}
\end{center}
\end{frame}

\begin{frame}
\begin{block}{Многозначная зависимость}
Пусть существует некоторое отношение $r$ со схемой $R$, а также два произвольных подмножества атрибутов $A,B\subseteq R$. 

Пусть $C = R \setminus (A\cup B)$.

В этом случае $B$ многозначно зависит от $A$, тогда и только тогда, когда множество значений атрибута $B$, соответствующее заданной паре $[a:A;c:C]$ отношения $r$, зависит от $a$ и не зависит от $c$. 
\end{block}
\begin{block}{4НФ}
Переменная отношения R находится в четвёртой нормальной форме, если она находится в НФБК и все нетривиальные многозначные зависимости фактически являются функциональными зависимостями от её потенциальных ключей. 
\end{block}
\end{frame}

\begin{frame}
Отношение \textbf{Дисциплины студентов}(Студент, Группа, Дисциплина)
\begin{center}
\includegraphics[scale=1.5]{images/ex-rasp-11.png}
\end{center}
\begin{itemize}
\item переменная отношения не соответствует 4НФ, так как существует следующая многозначная зависимость: 
\begin{itemize}
\item  \{Группа\} $\twoheadrightarrow$ \{Студент\}|\{Дисцилина\}
\end{itemize}
\item при добавлении нового студента в группу придется ему добавлять список изучаемых его группой дисциплин.
\end{itemize}
\end{frame}

\begin{frame}
Нормализованные отношения:
\textbf{Группы}(Группа, Студент)
\begin{center}
\includegraphics[scale=2]{images/ex-rasp-12.png}
\end{center}
\textbf{Дисциплины}(Группа, Дисциплина)
\begin{center}
\includegraphics[scale=2]{images/ex-rasp-13.png}
\end{center}
Если к исходной переменной отношения добавить атрибут, функционально зависящий от потенциального ключа, например оценку по дисциплине (Студент, Группа, Дисциплина) $\rightarrow$ Оценка, то полученное отношение будет находиться в 4НФ и его уже нельзя подвергнуть декомпозиции без потерь.
\end{frame}

\begin{frame}
\begin{block}{Декомпозицией отношения R}
называется замена R на совокупность отношений \{R1, R2,… , Rn\} такую, что каждое из них есть проекция R, и каждый атрибут R входит хотя бы в одну из проекций декомпозиции. 
\end{block}
Пример: для отношения R с атрибутами (a, b, c) существуют следующие варианты декомпозиции:
\begin{itemize}
\item (a), (b), (c)
\item (a), (b, c)
\item (a, b), (c)
\item (b), (a, c)
\item (a, b), (b, c)
\item (a, b), (a, c)
\item (b, c), (a, c)
\item (a, b), (b, c), (a, c)
\end{itemize}
\end{frame}

\begin{frame}
Рассмотрим отношение R', которое получается в результате операции естественного соединения (NATURAL JOIN), применённой к отношениям, полученным в результате декомпозиции R.
\begin{block}{Декомпозиция R называется декомпозицией без потерь}
если R' в точности совпадает с R. 
\end{block}
Пример: исходное отношение:
\begin{center}
\includegraphics[scale=0.5]{images/ex-rasp-14.png}
\end{center}
Результат декомпозиции:
\begin{center}
\includegraphics[scale=0.5]{images/ex-rasp-15.png}
\includegraphics[scale=0.5]{images/ex-rasp-16.png}
\end{center}
\end{frame}

\begin{frame}
Пример: исходное отношение:
\begin{center}
\includegraphics[scale=0.5]{images/ex-rasp-14.png}
\end{center}
Результат естественного соединения:
\begin{center}
\includegraphics[scale=0.5]{images/ex-rasp-17.png}
\end{center}
\end{frame}

\begin{frame}
Пример 2: исходное отношение:
\begin{center}
\includegraphics[scale=0.5]{images/ex-rasp-14.png}
\end{center}
Результат декомпозиции:
\begin{center}
\includegraphics[scale=0.5]{images/ex-rasp-15.png}
\includegraphics[scale=0.5]{images/ex-rasp-18.png}
\end{center}
Результат естественного соединения:
\begin{center}
\includegraphics[scale=0.5]{images/ex-rasp-14.png}
\end{center}
\end{frame}

\begin{frame}
\begin{block}{Декомпозиция R называется декомпозицией без потерь}
Пусть R — переменная отношения, а A, B, …, Z — некоторые подмножества множества её атрибутов.
Если декомпозиция любого допустимого значения R на отношения, состоящие из множеств атрибутов A, B, ... Z, является декомпозицией без потерь, говорят, что переменная отношения R удовлетворяет зависимости соединения *(А, В, . . . , Z)\end{block}
\begin{itemize}
\item понятие зависимости соединения определено не для отношения (конкретного значения), а для переменной отношения. \item зависимость соединения определяется не механически по текущим значениям, а следует из внешнего знания о природе и закономерностях данных, которые могут находиться в переменной отношения. 
\end{itemize}
\begin{block}{Зависимость соединения *\{A, B,…, Z\} является тривиальной}
когда по крайней мере одно из подмножеств A, B, ..., Z является множеством всех атрибутов отношения.
\end{block}
\end{frame}

\begin{frame}
\begin{block}{Отношение находится в 5НФ (в проекционно-соединительной нормальной форме)}
тогда и только тогда, когда каждая нетривиальная зависимость соединения в нём определяется потенциальным ключом (ключами) этого отношения
\end{block}
\begin{itemize}
\item зависимость соединения *\{A, B,…, Z\} определяется потенциальным ключом (ключами) тогда и только тогда, когда каждое из подмножеств A, B, ..., Z множества атрибутов является суперключом отношения. 
\item суперключ - подмножество атрибутов отношения, удовлетворяющее требованию уникальности: не существует двух кортежей данного отношения, в которых значения этого подмножества атрибутов совпадают (равны). 
\end{itemize}
\end{frame}

\begin{frame}
Пример 3: исходное отношение:
\begin{center}
\includegraphics[scale=0.5]{images/ex-rasp-19.png}
\end{center}
Результат декомпозиции:
\begin{center}
\includegraphics[scale=0.5]{images/ex-rasp-20.png}
\includegraphics[scale=0.5]{images/ex-rasp-21.png}
\includegraphics[scale=0.5]{images/ex-rasp-22.png}
\includegraphics[scale=0.5]{images/ex-rasp-23.png}
\end{center}
\end{frame}

\begin{frame}
\begin{block}{Отношение находится в 6НФ}
когда оно удовлетворяет всем нетривиальным зависимостям соединения (то есть не может быть подвергнуто дальнейшей декомпозиции без потерь)
\end{block}
\begin{center}
\includegraphics[scale=0.6]{images/ex-rasp-25.png}
\end{center}
Для хронологических баз данных максимально возможная декомпозиция позволяет бороться с избыточностью и упрощает поддержание целостности базы данных. 
\end{frame}

\begin{frame}
Исходное отношение:
\begin{center}
\includegraphics[scale=0.6]{images/ex-rasp-25.png}
\end{center}
Результат декомпозиции:
\begin{center}
\includegraphics[scale=0.6]{images/ex-rasp-26.png}
\end{center}
\begin{center}
\includegraphics[scale=0.6]{images/ex-rasp-27.png}
\end{center}
\end{frame}

\begin{frame}
\begin{itemize}
\item \textbf{теория нормализации является очень ценным достижением реляционной теории и практики}, поскольку она даёт научно строгие и обоснованные критерии качества проекта БД и формальные методы для усовершенствования этого качества;
\item \textbf{идеи нормализации полезны} для проектирования баз данных, они отнюдь \textbf{не являются универсальным или исчерпывающим средством} повышения качества проекта БД - существует слишком большое разнообразие возможных ошибок и недостатков в структуре БД, которые нормализацией не устраняются;
\item во всей сфере информационных технологий практически \textbf{отсутствуют методы оценки и улучшения проектных решений}, сопоставимые с \textbf{теорией нормализации реляционных баз данных} по уровню формальной строгости.
\end{itemize}
\end{frame} 

\end{document}